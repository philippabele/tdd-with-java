\chapter{Einleitung}

Die Notwendigkeit effektiver Aufgabenverwaltungslösungen ist in unserer zunehmend digitalisierten Welt unbestreitbar. Eine Todo-Liste ist ein bewährtes Werkzeug, das Einzelpersonen und Teams hilft, ihre täglichen Aufgaben zu organisieren, zu priorisieren und zu verfolgen. In einer Zeit, in der Effizienz und Produktivität entscheidend sind, bietet eine gut gestaltete Todo-Liste klare Vorteile bei der Strukturierung des Arbeitsalltags.

Ziel dieser Studienarbeit ist es, eine webbasierte Todo-Liste mit Test-Driven Development zu entwickeln, die Benutzern eine intuitive und effiziente Möglichkeit bietet, ihre Aufgaben zu verwalten. Die Anwendung soll es Einzelpersonen ermöglichen, Todos zu erstellen, zu bearbeiten und zu löschen und die Möglichkeit bieten eine Beschreibung, ein Fälligkeitsdaten, sowie das Markieren des Erledigen einer Todo hinzuzufügen. Darüber hinaus soll die Todo-Liste Funktionen wie Benutzerregistrierung und -authentifizierung bereitstellen, um die Sicherheit und Privatsphäre der Benutzerdaten zu gewährleisten.

Im Rahmen dieser Entwicklung werden zwei Hauptkomponenten erstellt: das Frontend, das die Benutzeroberfläche der Todo-Liste bereitstellt, und das Backend, das die Datenverarbeitung und -speicherung übernimmt. Die Anwendung wird auf modernen Technologien basieren, darunter Vue.js für das Frontend, Spring Boot für das Backend, MySQL als relationale Datenbank.


Es werden grundlegende Kenntnisse in HTML, CSS, JavaScript, sowie in der Entwicklung mit Vue.js, Spring Boot, MySQL vorausgesetzt.

Diese Dokumentation bietet einen umfassenden Einblick in die Architektur, Implementierung und Teststrategie der Todo-Liste Webanwendung.